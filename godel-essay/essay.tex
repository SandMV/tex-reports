\documentclass[specialist,
               substylefile = spbu.rtx,
               subf,href,colorlinks=true, 12pt]{disser}

\usepackage[a4paper,
            mag=1000, includefoot,
            left=3cm, right=1.5cm, top=2cm, bottom=2cm, headsep=1cm, footskip=1cm]{geometry}
\usepackage[T2A]{fontenc}
\usepackage[utf8]{inputenc}
\usepackage[english,russian]{babel}
\usepackage{xcolor}
\usepackage{amsmath,amssymb,amsfonts,dsfont}
\usepackage{algorithmic}
\usepackage{graphicx,wrapfig}
\ifpdf\usepackage{epstopdf}\fi

%\setcitestyle{semicolon}

\let\vec=\mathbf

\setcounter{tocdepth}{2}

\graphicspath{{figures/}}

\newcommand{\todo}[1]{ {\color{red} TODO: {#1}}}
\newcommand{\figref}[1]{(Рис. \ref{#1})}
\renewcommand{\it}[1]{{\textit{#1}}}

%\makeatletter

%\newcommand\float@endH{\@endfloatbox\vskip\intextsep
 % \if@flstyle\setbox\@currbox\float@makebox\columnwidth\fi
  %\box\@currbox\vskip\intextsep\relax\@doendpe}

%\makeatother
%\renewcommand{\todo}[1]{}

\DeclareMathOperator{\sign}{sign}

\newtheorem{theorem}{Теорема}

\begin{document}

\institution{
    Санкт-Петербургский государственный университет \\
    Прикладная математика и информатика \\
    Статистическое моделирование
}

\title{
Реферат}

\topic{\normalfont\scshape
Курт Гёдель. Теорема <<О неполноте>>}

\author{ Сандул Михаил Вадимович }

%\sa       { Шпилёв П.В.}
%\sastatus { к.ф.-м.н. }

\city{ Санкт-Петербург }
\date{\number\year}

\maketitle

\chapter{Биография\cite{ias}\cite{st-andr}\cite{kreisel}\cite{stanford}}

Курт Гёдель --- выдающийся математик XX века. Он родился 28 апреля 1906 года в городе Брюнне в Моравии, в Австро-Венгерской империи. После первой мировой войны этот город отошел к Чехословакии и стал называться Брно.\par

Рудольф Гёдель, отец Курта, родился в Вене. В Брюнне он стал совладельцем и управляющим в одной из крупнейших текстильных компаний, поэтому его семья жила с комфортом, и могла позволить себе содержать прислугу и гувернанток для детей: Курта и его старшего брата Рудольфа.\par

Мать Курта, Марианна Хандшух, была родом из Рейнской области. Она получила гуманитарное образование, а часть её школьных занятий проходила во Франции.\par

Гёдель рос, в общем, счастливым, но застенчивым мальчиком: например, он сильно огорчался, если проигрывал или когда мать уходила из дома. В шесть лет у Гёделя случился мучительный приступ ревматизма, но когда он прошел, возобновилась нормальная жизнь. Чуть позже, в восемь лет, видимо, прочтя в какой-то медицинской книге о возможных осложнениях после болезни, Гедёль уверил себя, что у него слабое сердце. С тех пор в его жизни озабоченность здоровьем стала занимать все большее место, так, например, Курт Гёдель даже в жаркую погоду ходил в теплой одежде.\par

Впрочем, большая часть его жизни прошла благополучно. Ещё со школы Гёдель проявлял способностм к сосредоточенной работе и концентрации внимания. Его брат, Рудольф, рассказывал, что у Курта в школьные годы был относительно ограниченный круг интересов, и, к восхищению учителей и одноклассников, к концу учебы в гимназии он уже освоил университетский курс математики. Он добавляет, что кроме математики, Курт Гёдель увлекался языками, и говорят, что за все время учебы, по всем заданиям по латинскому языку Курт получил наивысшие оценки и не допустил ни единой грамматической ошибки.\par

В 1923 году Курт Гёдель поступил в университет Вены. Более всего его интересовала теоретическая физика и математика, но позже он решил посвятить себя именно математике. Его выбор определил трехгодичный курс теории чисел, прочитанный Филиппом Фуртвенглером, братом знаменитого немецкого дирижера Вильгельма Фуртвенглера. Филипп Фуртвенглер был парализован от шеи и ниже, поэтому читал лекции в инвалидном кресле без заранее подготовленного текста, а доказательства на доске записывал помощник.  Его занятия производили впечатления на любого студента, но для Гёделя, который очень переживал о своём здоровье, они имели особенное значение.\par

Будучи студентом, К. Гёдель посещал собрания, которые проводил Мориц Шлик, профессор университета. На этих собраниях в основном обсуждались вопросы связанные с языком и значением логических отношений, например таких как логическое следствие, а также обсуждалась работа Бертрана Рассела <<Введение в математическую философию>>. Участники этих собраний образовали <<Венский кружок>> (Der Wiener Kreis) --- группу философов позитивистов. Эти встречи оказали большое влияние на Курта, и уже после миграции в Америку, К. Гёдель больше погрузится в философию и напишет несколько работ, связанных с ней.\par

Однако, главным учителем Гёделя был аналитик Ханс Хан, который также принимал участие в собраниях <<Венского кружка>>. Именно под руководством Ханса Хана Гёдель написал докторскую диссертацию в 1929 году, а в 1930 получил место на кафедре в венском университете. Свою самую известную работу <<О неполноте>> К. Гёдель опубликовал в 1931 году, а в 1932 году Х. Хан принял её в качестве диссертации на звание доцента, как намного превышающую необходимые требования. В марте 1933 года Гёдель стал приват-доцентом (неоплачиваемый лектор) в Венском университете.\par

В 1938 году Австрия вошла в состав Германии, и должность приват-доцента была упразднена. Большинство преподавателей, которые занимали такую должность, стали оплачиваемыми лекторами (Dozent neuer Ordnung), но Гёделю в этой должности отказали: его считали евреем. По этой же причине на него однажды напали какие-то хулиганы во время прогулки с женой.\par

После начала войны в 1939 году, Курт Гёдель стал опасаться призыва в армию, и в конце этого же года Курт вместе со своей женой Адель уехали из Австрии в Америку, проехав через Россию по транссибирской железной дороге. В марте 1940 учёный прибыл в Сан Франциско.\par

В Америке Гёдель вместе с женой обосновались в Принстоне. Свою работу он продолжил в Институте Перспективных Исследований в качестве ординарного члена до 1946 года, а с 1946 --- в качестве постоянного члена. В 1948 году Гёдель получил американское гражданство. В 1953 году стал профессором в Институте Перспективных Исследований. Он проработал в этой должности до ухода на пенсию в 1976 году.\par

В первые годы работы в Институте Курт Гёдель близко сошелся с Альбертом Энштейном. 1940-е ещё примечательны тем, что он углубился в философию математики. В эти года учёный написал несколько важных работ на эту тему: <<О математической логике Рассела>> в 1944, <<Что такое континуум гипотеза Кантора?>> в 1947, а в 1949 опубликовал <<Замечание о связи теории относительности и идеалистической философии>>. Разумеется, он продолжил исследования в этой области и дальше. 

Гёдель умер в Принстоне в 14 января 1978 в 71 год. Его жена Адель пережила его на три года.

\chapter{Теоремы о неполноте}

Ещё в начале 20 века Давид Гильберт провозгласил цель аксиоматизировать всю математику, и для завершения этой задачи оставалось доказать непротиворечивость и логическую полноту арифметики натуральных чисел. Ещё в 1929 году в своей диссертации Гёдель указывает на возможную не разрешимость поставленной задачи.\par

Первая теорема о неполноте доказывает неполноту системы аксиом Пеано, приводя пример формулы, которая не может быть ни доказана, ни отклонена как несправедливая. В исходной формулировке своей первой теоремы о неполноте, Курт Гёдель использовал понятие $\omega$-непротиворечивости формальной системы и свою же теорему <<О неподвижной точке>> для доказательства неполноты. Позже эта формулировка была упрощена Россером, и мы приведем именно её:

\begin{theorem}
Первая теорема о неполноте. Если формальная система $S$ непротиворечива, то в ней невозможно вывести обе формулы $B$ и $\neg B$. Иначе говоря, если система $S$ непротиворечива, то она неполна, и $B$ служит примером неразрешимой формулы.
\end{theorem}

Вторая теорема о неполноте доказывает недоказуемость инструментами теории чисел непротиворечивость теории чисел. 

\begin{theorem}
Вторая теорема о неполноте. Если формальная арифметика $S$ непротиворечива, то в ней невозможно вывести формулу, содержательно утверждающую непротиворечивость $S$.
\end{theorem}

\bibliographystyle{gost2008}

\bibliography{biblio}

\end{document}