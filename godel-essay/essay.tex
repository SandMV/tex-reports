\documentclass[specialist,
               substylefile = spbu.rtx,
               subf,href,colorlinks=true, 12pt]{disser}

\usepackage[a4paper,
            mag=1000, includefoot,
            left=3cm, right=1.5cm, top=2cm, bottom=2cm, headsep=1cm, footskip=1cm]{geometry}
\usepackage[T2A]{fontenc}
\usepackage[utf8]{inputenc}
\usepackage[english,russian]{babel}
\usepackage{xcolor}
\usepackage{amsmath,amssymb,amsfonts,dsfont}
\usepackage{algorithmic}
\usepackage{graphicx,wrapfig}
\ifpdf\usepackage{epstopdf}\fi

%\setcitestyle{semicolon}

\let\vec=\mathbf

\setcounter{tocdepth}{2}

\graphicspath{{figures/}}

\newcommand{\todo}[1]{ {\color{red} TODO: {#1}}}
\newcommand{\figref}[1]{(Рис. \ref{#1})}
\renewcommand{\it}[1]{{\textit{#1}}}

%\makeatletter

%\newcommand\float@endH{\@endfloatbox\vskip\intextsep
 % \if@flstyle\setbox\@currbox\float@makebox\columnwidth\fi
  %\box\@currbox\vskip\intextsep\relax\@doendpe}

%\makeatother
%\renewcommand{\todo}[1]{}

\DeclareMathOperator{\sign}{sign}

\begin{document}

\institution{
    Санкт-Петербургский государственный университет \\
    Прикладная математика и информатика \\
    Статистическое моделирование
}

\title{
Реферат}

\topic{\normalfont\scshape
Курт Гёдель. Теорема <<О неполноте>>}

\author{ Сандул Михаил Вадимович }

%\sa       { Шпилёв П.В.}
%\sastatus { к.ф.-м.н. }

\city{ Санкт-Петербург }
\date{\number\year}

\maketitle

\chapter{Биография}

Курт Гёдель --- выдающийся математик XX века. Он родился 28 апреля 1906 года в городе Брюнне в Моравии, в Австро-Венгерской империи. После первой мировой войны этот город отошел к Чехословакии и и стал называться Брно. Рудольф Гёдель, отец Курта, работал управляющим в одной из текстильных фирм. Мать Гёделя получила широкое гуманитарное образование.\par

Гёдель рос, в общем, счастливым, но застенчивым мальчиком: например, он сильно огорчался, если проигрывал или когда мать уходила из дома. В шесть лет у Гёделя случился мучительный приступ ревматизма, но когда он прошел, возобновилась нормальная жизнь. Чуть позже, в восемь лет, видимо, прочтя в какой-то медицинской книге о возможных осложнениях после болезни, Гедёль уверил себя, что у него слабое сердце. С тех пор в его жизни озабоченность здоровьем стала занимать все большее место. \todo{Гедель даже в жаркую погоду ходил  в теплой одежде --- эта причуда и есть одно из следствий его озабоченности здоровьем}\par

Впрочем, большая часть его жизни прошла благополучно. Ещё со школы Гёдель проявлял способность к сосредоточенной работе и концентрации внимания, где он добился репутации безупречного знатока латинского языка. Позже он колебался между математикой и теоретической физикой. Его решение определил трехгодичный курс теории чисел, прочитанный П. Фуртвенглером. Однако главным учителем Гёделя был аналитик Г. Хан. Именно Г. Хан 1 декабря 1932 года принял в каче
стве диссертации на звание доцента работу Гёделя о неполноте.\par

Когда Гёделю был 21 год, он встретил встретил Адель Поркерт в венском ночном клубе. Она была на шесть лет старше и до их встречи недолгое время была замужем. Родители Курта возражали против такого выбора, и Гёдель женился на Адели только в 1938 году.\par



В марте 

Гёдель 
\end{document}